mepigraph{Reductio ad absurdum, which Euclid loved so much, is one of a mathematician's finest weapons. It is a far finer gambit than any chess play: a chess player may offer the sacrifice of a pawn or even a piece, but a mathematician offers the game.}{G. H. Hardy (A Mathematician's Apology, 1940)}
\par
\section{Introduction}
\begin{definition} \label{def:rings}
  $( R, +, * )$ is a ring if:
  \begin{itemize}
    \item $R$ is a commutative group under $+$.
    \item $R$ is closed under $*$, associative and has identity.
    \item For all $a, b \in R$ $a * (b + c) = a * b + a * c$ and similarly $(b+c) * a = b * a + c * a$
  \end{itemize}
\end{definition}
If $*$ is commutative, $R$ is called a \textbf{commutative ring}.

\begin{definition} \label{def:fields}
  $( F, +, * )$ is a field if:
  \begin{itemize}
    \item $F$ is a commutative group under $+$.
    \item $F^{*} = F \setminus \{ 0 \}$ is a commutative group under $*$.
    \item Distributivity property holds.
  \end{itemize}
\end{definition}
Fields will be studied more in detail in the sebsequent chapter.

\subsection{Examples}
\begin{itemize}
  \item \emph{Example 1:} $\mathbb{Z}$ is a commutative ring.
  \item \emph{Example 2:} $\mathbb{Q}, \mathbb{R}, \mathbb{C}$ are all fields.
  \item \emph{Example 3:} $\mathbb{Z}_{n}$ is a ring.
  \item \emph{Example 4:} $\mathbb{Z}_{p}$ is a field, when $p$ is prime.
  \item \emph{Example 5:} $n \times n$ matrices from $\mathbb{Q}$ is a ring.
  \item \emph{Example 6:} For any field $F$, vectors in $F^{n}$ form a commutative ring, under componentwise operations.
  \item \emph{Example 7:} $R[x]$, the set of all polynomials in $x$ with coefficients from a ring $R$, is a ring under polynomial addition and multiplication. This can be extended to multivariate polynomials over $R$.
  \item \emph{Example 8:} Let $\mathcal{F}$ be the set of all continuous maps from $\mathbb{R} \rightarrow \mathbb{R}$.
    $\mathcal{F}$ is a commutative ring, under the operations $+, *$ defined as $(f + g)(x) = f(x) + g(x)$ and $(f * g)(x) = f(x) * g(x)$ for all $f, g$.
\end{itemize}

\section{Units in Rings}
Note that from this point onwards, for ring elements $a, b$, we will use $ab$ as shorthand for $a*b$.
\begin{definition} \label{def:units}
  Let $R$ be a commutative ring.
  $a \in R$ is a unit if there exists an element $b$ in $R$ such that $ab = 1$.
\end{definition}

\subsection{Examples}
Some very trivial examples of units are $\pm 1$ in $\mathbb{Z}$ and $\mathbb{Z}^{*}$ in $\mathbb{Z}_{n}$.
\par
A more interesting example follows.
Consider $\mathbb{Z}[\sqrt{2}]$, which are the set of all polynomials, with $\sqrt{2}$ substituted for $x$.
Elements of this ring are of the form $a_{0} + a_{1} \sqrt{2} + \dots + a_{i} \sqrt{2}^{i}$ which can be easily generalized as $\{ a + \sqrt{2} b | a, b \in \mathbb{Z} \}$.
For some $a+ \sqrt{2} b$ to be an unit, we require $(a + \sqrt{2}b)(c + \sqrt{2} d) = 1$ for some $c, d$.
This gives us $ac + 2bd + \sqrt{2} (ad + bc) = 1 \Rightarrow ad = -bc \Rightarrow d = -\frac{bc}{a}$
Substituting this back gives us $a^{2} - 2b^{2} = \frac{a}{c}$.
Since $a, b, c, d$ are integers, the above can only hold when $a = \alpha c$ and $b = -\alpha d$ for some $\alpha$.
This finally gives us $1 = (\alpha c - \sqrt{2} \alpha d)(c + \sqrt{2} d) \Rightarrow \alpha (c^{2} - 2d^{2}) = 1 \Rightarrow \alpha = \pm 1, (c^{2} - 2 d^{2}) = \pm 1$.
The solutions to the above equation give us all units, and it can be shown that the only possible units are the powers of $(1 + \sqrt{2})$.
\section{Defining primes in $R$}
We now try and define primes in rings.
\par
The first definition we can try is that $a \in R$ is prime if whenever $a = bc$, then wither $b$ or $c$ are units.
This definition is clearly just an extension of the definition of primes over integers, but is not very adequate, since some rings, like $\mathbb{Z}_{n}$ have no primes under this definition, except the units.
\par
\emph{Definition 2:} $a \in R$ is prime if whenever $a$ divides $bc$, it either divides $b$ or $c$.
Consider $\mathbb{Z}_{6}$.
$2$ can be written as $2 = 2*4, 2*1, 4*5$, and it is easily checked that $2$ always divides one of the elements of the product.
Thus, $2$ is a prime under this definition.
\par
$a \in R$ is defined as \textbf{irreducible} if whenever $a=bc$, either $b$ or $c$ is a unit, and $a$ is not.
\section{Solutions of $y^{3} = x^{2} + 2$}
We now use the tools developed so far to try and find the solutions of this equation.
The proof here is just a rough sketch, and the details can be easily filled out.
\par
We can factorize $x^{2} + 2 = (x + i \sqrt{2})(x - i \sqrt{2})$.
Consider $\mathbb{Z}[i \sqrt{2}]$, which has elements of the form $ \{ a + i \sqrt{2} b | a, b \in \mathbb{Z} \}$.
If we assume that $y, (x + i \sqrt{2}), (x - i \sqrt{2})$ are all irreducible, then there cannot be any solutions.
Thus $y$ needs to be reducible.
\par
Starting with the assumption that $y = (a + i \sqrt{2} b)(c + i \sqrt{2} d)$ gives us that $y$ needs to be of the form $y = (m + i \sqrt{2} n)(m - i \sqrt{2}n)$ for some $m, n$.
These are called conjugates of each other.
\par
We have $(m - i \sqrt{2}n)^{3}(m + i \sqrt{2} n)^{3} = (x + i \sqrt{2})(x - i \sqrt{2})$.
Whatever we equate for $(x + i \sqrt{2})$, we need to assign to $(x - i \sqrt{2})$ its conjugate.
This gives us two cases, $(x + i \sqrt{2}) = (m + i \sqrt{2} n)^{3}$ and $(x + i \sqrt{2}) = (m + i \sqrt{2} n)^{2}(m - i \sqrt{2}n)$.
The remaining two cases are similar to these, and are ignored here.
\par
\emph{Case I:} gives us $(x + i \sqrt{2}) = (m + i \sqrt{2} n)^{3} = m^{3} - 6n^{2}m + (3m^{2}n - 2 n^{3}) i \sqrt{2}$ which further gives us $x = m^{3} - 6 n^{2}m$ and $1 = 3m^{2}n - 2 b^{3} = 1$.
Applying the constraint that $m, n$ are integers, we get $n = 1$ and $a = \pm 1$ and thus $x = \pm 5$.
\par
\emph{Case II:} gives us $x = m(m^{2} + 2n^{2})$ and $1 = n(m^{2} + 2n^{2})$.
The integer solutions again give us $n = \pm 1$ which then gives $m^{2} + 2 = \pm 1$, which has no solutions.
\par
This shows that solution to the equation include $x = \pm 5, y = 3$.

\section{Ideals}
In this section, a number of lemma and definition are presented.
\par
Consider the ring $\mathbb{Z}[i\sqrt{5}]$.
In this ring, $9 = 3 \times 3 = (2 + i \sqrt{5})(2 - i \sqrt{5})$.
$3, (2 + i\sqrt{5}), (2 - i\sqrt{5})$ are all irreducible, therefore, we cannot cancel anything on either side.
$\mathbb{Z}[i \sqrt{5}]$ is said to not admit unique factorization.
\subsection{Division in Rings}
Ring elements satisfy the following two properties:
\begin{itemize}
  \item For all $a, b, c \in R$ if $a | b$ and $a | c$ then $a | bc$.
  \item If $a | b$ then $a | bc$.
\end{itemize}
\subsection{Ideal Numbers and Ideals}
Attempts to fix the problem of Unique Factorizing led to the development of ideal numbers, which generalized to ideals.
\par
We represent elements $a \in Ra$ by all $b$ such that $a | b$.
Let all the $b$ for a given $a$ be represented by $(a)$.
\begin{lemma} \label{lem: idealnumbers}
  If $(a) = (c)$ then $a = c \times u$ for some unit $u$.
\end{lemma}
\begin{definition}\label{def:ideals}
  $I \subseteq R$ is an ideal if the following hold:
  \begin{itemize}
    \item $(I, +)$ is a subgroup of $(R, +)$.
    \item For all $a \in I$ and $b \in R$,  $ab \in I$.
  \end{itemize}
\end{definition}
In the ring $\mathbb{Z}[ i \sqrt{5}]$, some ideals are $(3), (2 + i \sqrt{5}), (2 - i \sqrt{5})$.
Further, consider $(3, 2 + i \sqrt{5}) = \{ 3a + (2 + i \sqrt{5}) b | a, b \in R \}$.
The fact that this is an ideal is quite clear.
\par
Given ideal $I$ of a ring, and a subset $S$ of $I$, $S \subseteq I$, $I$ is generated by $S$, if for every $a \in I, a = \Sigma \alpha_{i} \beta_{i}$, for $\alpha_{i} \in R$ and $\beta_{i} \in S$.
Elements of $S$ are called generators of $I$.
$S$ can be either finite or infinite.
\par
Ideal $I$ is called a \textbf{principal ideal} if it can be generated by a single element.
\par
Let $I_{1}, I_{2}$ be ideals of $R$.
Then $I_{1} * I_{2} = \{ \Sigma a_{i} b_{i} | a_{i} \in I_{1}, b_{i} \in I_{2} \}$
\begin{lemma} \label{lem:idealproduct}
  $I_{1} * I_{2}$ is an ideal.
\end{lemma}
\begin{proof}
  If $c \in I_{1} * I_{2}$ and $\alpha \in R$, then $\alpha c = \alpha \Sigma a_{i} b_{i} = \Sigma (\alpha a_{i}) b_{i}$.
\end{proof}
\begin{definition} \label{def:primeideal}
  Ideal $I$ is called \textbf{prime} if for every $I_{1}, I_{2}$, $I = I_{1} * I_{2} \Rightarrow I_{1} = (1)$ or $I_{2} = (1)$.
\end{definition}
\begin{lemma} \label{lem:identityideal}
  For any ideal $I, I * (1) = I$.
\end{lemma}
\begin{proof} \label{proof:identityideal}
  $a \in I \Rightarrow a = a \times 1 \in I * (1)$ \\
  $a \in I * (1) \Rightarrow a = \Sigma a_{i} b_{i}$.
  $a_{i} \in I, b_{i} \in R$ by properties of ideals.
  This implies $a_{i} b_{i} \in I \Rightarrow \Sigma a_{i} b_{i} \in I$.
\end{proof}
\subsubsection{Examples in $\mathbb{Z}[i \sqrt{5}]$ }
In $\mathbb{Z}[ i \sqrt{5}], (3) = (3, 2 + i \sqrt{5}) * (3, 2 - i \sqrt{5})$.
\begin{proof}
  First we need to show that $3 \in (3, 2 + i \sqrt{5}) * (3, 2 - i \sqrt{5})$.
  Consider $3 * (2 - i \sqrt{5}), 3 * (2 + i \sqrt{5}), -(2 - i \sqrt{5}) * (2 + i \sqrt{5})$ all that belong to $(3, 2 - i \sqrt{5}) * (3, 2 + i \sqrt{5})$.
  Their sum in $3$.
  \par
  Second, consider an element of $(3, 2 - i \sqrt{5}) * (3, 2 + i \sqrt{5}), (3a + (2 + i \sqrt{5})b) * (3c + (2 - i \sqrt{5})d) = 9ac + 9bd + 3a(2 - i \sqrt{5}) + 3c(2 + i \sqrt{5}) \in (3).$
\end{proof}
Similarly, in $\mathbb{Z}[i \sqrt{5}], (2 + i \sqrt{5}) = (3, 2 + i \sqrt{5}) * (3, 2 + i \sqrt{5})$ and $(2 - i \sqrt{5}) = (3, 2 - i \sqrt{5}) * (3, 2 - i \sqrt{5})$.
\subsection{Examples of Ideals in other Rings}
\begin{itemize}
  \item \emph{Example 1:} In $\mathbb{Z}$, $(2), (3) ... $ are the principal ideals.
    There are no other ideals, since for any $a, b \in \mathbb{Z}$, we have $(a, b) = (gcd(a, b))$.
    This is given by the fact that the gcd of any two numbers is an integer linear combination of the numbers.
  \item \emph{Example 2:} In $\mathbb{Z}_{n}, (m) = \mathbb{Z}_{n}$ if $gcd(m, n) = 1$.
    In $\mathbb{Z}_{6}, (2) = \{ 0, 2, 4 \}, (3) = \{0, 3 \} $
  \item \emph{Example 3:} In $\mathbb{Z}[x]$, the ideal $(p(x))$ for any polynomial $p(x)$ is just all its multiples.
  \item \emph{Example 4:} In $\mathbb{Z}[x, y]$, consider the ideal $(x, y).$
    It consists of all polynomals that have $0$ as the constant term.
  \item \emph{Example 5:} Consider the polynomial generated by $(x, y)$ in $\mathbb{Q}[x, y]$.
    This ideal has the property that the only ideal larger than it is $(1)$.
    \begin{proof} \label{pf:maxideals}
      Consider an ideal $I$ such that $(x, y) \subsetneq I.$
      Then $p(x, y) = c + q(x, y)$ such that $q(x, y) \in (x, y), c \neq 0$.
      This gives us $c = p(x, y) - q(x, y) \in I \Rightarrow 1 \in I \Rightarrow I = (1)$.
    \end{proof}
    \begin{definition}
      Ideal $I$ is \textbf{maximal} if $(1)$ is the only ideal larger than $I$.
    \end{definition}
  \item \emph{Example 6:} Consider the ring $C_{x}$, the set of all continuous functions from $\mathbb{R} \rightarrow \mathbb{R}$.
    $I = \{ f \in C_{x}, f(0) = 0 \}$.
\end{itemize}

\section{Ring Properties, Morphisms and Quotienting}
\begin{definition} \label{def:dedekind}
  Ring $R$ is called a \textbf{Dedekind Domain} if
  \begin{itemize}
    \item $R$ is an integral domain. (If $a*b = 0$ then $a = 0$ or $b = 0$)
    \item $R$ is integrally closed.
    \item All prime ideals of $R$ are maximal.
    \item All ideals of $R$ are finitely generated
  \end{itemize}
\end{definition}
\begin{theorem} \label{the:dedekind}
  If $R$ is a Dedekind Domain, then every ideal $I$ of $R$ can be uniquely written as a product of prime factors.
\end{theorem}
\begin{definition} \label{def:homomorphism}
  Let $R_{1}$ and $R_{2}$ be two rings.
  Function $\phi : R_{1} \rightarrow R_{2}$ is a ring homomorphism if
  \begin{itemize}
    \item For all $a, b \in R_{1}, \phi(a + b) = \phi(a) + \phi(b)$
    \item For all $a, b \in R_{1}, \phi(a * b) = \phi(a) * \phi(b)$
  \end{itemize}
\end{definition}
The above two properties lead to the following:
\par
$\phi(0) = \phi(0 + 0) = \phi(0) + \phi(0) \Rightarrow \phi(0) = 0$.
\par
$\phi(1) = \phi(1 * 1) = \phi(1) * \phi(1) \Rightarrow \phi(1) * (\phi(1) - 1) = 0$.
This implies that $\phi(1) = 1$ or $\phi(1) = 1$.
If $\phi(1) = 0$ then $\phi(a) = \phi(1 * a) = \phi(1) * \phi(a) = 0, \forall a$.
\begin{definition} \label{def:ringker}
  The kernel of $\phi$ is $A \subseteq R$ such that $\phi(a) = 0$ for all $a, \in A$.
  It is represented as $\ker(\phi)$.
\end{definition}
$ker(\phi)$ is an additive subgroup of $R$.
Further, if $a \in ker(\phi)$ then we have $\phi(a * b) = \phi(a) * \phi(b) = 0 * \phi(b) = 0 \Rightarrow ker(\phi)$ is an ideal of the ring.
\par
We can further define equivalence classes in $R$, induced by $\phi$, as we did for groups.
The classes are represented by $ker(\phi), ker(\phi) + a_{1}, ker(\phi) + a_{2} \dots $ for elements $a_{1}, a_{2} \not \in ker(\phi)$.
\par
In general, for an ideal $A$, we have $R \big/ A = \{ A, a_{1} + A, a_{2} + A, \dots \}$.
\begin{lemma} \label{lem:quotient}
  $R \big/ A$ is a ring where $(a_{1} + A) + (a_{2} + A) = (a_{1} + a_{2}) + A$ and $(a_{1} + A) * (a_{2} + A) = (a_{1} * a_{2}) + A$.
\end{lemma}
\begin{proof} \label{proof:quotient}
  $ R \big/ A$ is a commutative group under $+$, because both $R$ and $A$ are commutative groups under addition.
  Further considering multiplication, $(a_{1} + \alpha_{1}) * (a_{2} + \alpha_{2}) = a_{1} * a_{2} + a_{1} * \alpha_{2} + \alpha_{1} * \alpha_{2} + a_{2} + \alpha_{1}$.
  We have $a_{1} * \alpha_{2}, a_{2} + \alpha_{1}, \alpha_{1} + \alpha_{2} \in A$ which means $(a_{1} + \alpha_{1}) * (a_{2} + \alpha_{2}) \in (a_{1} + a_{2}) + A$ independant of the representative element.
  Finally, distributivity is clear from the distributive property of $R$.
\end{proof}
Based on whether we quotient by a maximal ideal, prime ideal or principal ideal, $R \big/ A$ has different properties.
\subsection{Quotienting by maximal ideals}
\begin{lemma} \label{lem:maximalquotient}
  When $I$ is a maximal ideal, $R \big/ A$ is a field.
\end{lemma}
\begin{proof} \label{proof:maximalquotient}
  Let $a + I \in R \big/ I, a \neq 0$.
  Further, let $0 = I, 1 = 1 + I$.
  We need to find the inverse of $R \big/ I$.
  Define $J = (a, I)$, the ideal containing both $a$ and $I$.
  $I \subsetneq J$ since $a \not \in I$.
  Since $I$ is maximal, $J = (1)$
  The elements of $J$ are $a * b + \alpha$ for $b \in R, \alpha \in I$.
  Since $1 \in J, 1 = a * b + \alpha$.
  In $R \big/ I$ we get $(a + I) * (b + I) = (a * b) + I = 1 + I \Rightarrow (a + I)^{-1} = (b + I)$.
\end{proof}
\subsubsection{Defining real numbers using maximal ideals}
We now define real numbers using quotienting using maximal ideals.
\begin{definition} \label{def:cauchy}
  $(a_{0}, a_{1}, a_{2}, \dots ), a_{i} \in R$ is a \textbf{Cauchy Sequence} if for all $\epsilon > 0, \epsilon \in \mathbb{Q}, \exists m > 0, m \in \mathbb{Z}$ such that for all $ n \geq m, n \in \mathbb{Z}, |a_{n} - a_{m} | < \epsilon$.
\end{definition}
For example, $ \{ \frac{3}{1}, \frac{31}{10}, \frac{314}{100}, \frac{3141}{1000}, \frac{31415}{10000}, \dots \}$ is a cauchy sequence that converges to $\pi$.
\begin{theorem} \label{theorem:cauchyring}
  Let $\mathcal{R}$ denote the set of all cauchy sequences.
  $\mathcal{R}$ is a ring under componentwise addition and multiplication.
\end{theorem}
\begin{proof} \label{proof:cauchyring}
  Associativity and identity are borrowed from the ring of real numbers.
  We need to show closure.
  \par
  Let $S_{1} = (a_{0}, a_{1}, a_{2}, \dots )$ and $S_{2} = (b_{0}, b_{1}, b_{2}, \dots ).$
  Fix an $\epsilon > 0$.
  Let $m_{1} \in \mathbb{Z}$ such that $|a_{n} - a_{m_{1}}| < \frac{\epsilon}{4} \forall n \geq m_{1}$.
  Similarly $m_{2} \in \mathbb{Z}$ such that $|b_{n} - b_{m_{2}}| < \frac{\epsilon}{4} \forall n \geq m_{2}$
  Let $m = max \{ m_{1}, m_{2} \}$.
  \par
  We have $|a_{n} + b_{n} - a_{m} - b_{m}| \leq |a_{n} - a_{m}| + |b_{n} - b_{m}|, \forall n > m$.
  Further, $|a_{n} - a_{m}| + |b_{n} - b_{m}| \leq |a_{n} - a_{m_{1}}| + |a_{m_{1}} - a_{m}| + |b_{n} - b_{m_{2}}| + |b_{m_{2}} - b_{m}| \leq \frac{\epsilon}{4} + \frac{\epsilon}{4} + \frac{\epsilon}{4} + \frac{\epsilon}{4} \leq \epsilon$.
  Therefore, $S_{1} + S_{2}$ is cauchy.
  \par
  Similarly, by picking the right $m$, we can prove that $S_{1} * S_{2}$ is also cauchy.
  This has been left as an exercise.
\end{proof}
Let $I$ be the set of all cauchy sequences converging to $0$.
\begin{theorem} \label{theorem:cauchymax}
  $I$ is a maximal ideal of $\mathcal{R}$.
\end{theorem}
\begin{proof} \label{proof:cauchymax}
  $I$ is clearly a commutative group under addition.
  Further, for any $S \in I$, $S' \in \mathcal{R}$, $S*S'$ clearly converges to $0$.
  \par
  Let $S' = (b_{0}, b_{1}, \dots ) \in \mathcal{R}$ such that $S' \not \in I$.
  Since $S' \not \in I$, there exists $\delta > 0$ such that for all $n \geq n_{\circ}$ for some $n_{\circ}$ we have $|b_{n}| > \delta$.
  Let $J$ be the ideal containing $I$ and $S'$ and $k = max \{ |b_{i} | \}, 0 \leq i \leq n_{\circ}$
  Let $t$ be the cauchy sequence with $k + \delta$ as the first $n_{\circ}$ elements, and $0$ as the rest of the elements.
  $t \in I \Rightarrow S' + t \in J$.
  All elements of $S' + t = (c_{0}, c_{1}, \dots )$ are atleast as big as $\delta$ in magnitude.
  Define $t' = (\frac{1}{c_{0}}, \frac{1}{c_{1}}, \dots )$.
  $t'$ is cauchy, since $| \frac{1}{c_{n}} - \frac{1}{c_{m}}| = \frac{|c_{m} - c_{n}|}{|c_{m} c_{n}|} \leq \frac{|c_{n} - c_{m}|}{\delta^{2}} \leq \frac{\epsilon}{\delta^{2}}$, since $S' + t$ is cauchy.
  \par
  Therefore, we have $t' \in \mathcal{R}$ and thus $t' * (S' + t) \in J \Rightarrow (1, 1, \dots) \in J \Rightarrow J = \mathcal{R}$.
\end{proof}
$\mathbb{R} = \mathcal{R} \big/ {I}$ is thus a field, and it is exactly the field of real numbers.
This is because, there is exactly one sequence in it converging to every real number.
\par
The above exercise can be repeated with different valuations to get fields such as the the field of p-adics.
\subsection{Other Examples of Ring Quotienting}
\begin{itemize}
  \item \emph{Example 1:} Consider the ring $\mathbb{Z}[i \sqrt{5}]$ and the ideal $(3, 2 + i \sqrt{5})$.
    The ring is a dedekind domain, therefore $(3, 2 + i \sqrt{5})$, which is prime, is also maximal.
    An arbitrary element of the ring is $a + i \sqrt{5}b = (a - 2b) + (2 + i \sqrt{5})b$.
    $a - 2b$ is an integer, and thus can be written as $\gamma + 3c$ where $\gamma$ is either $0, 1$ or $2$.
    Thus, an arbitrary element is of the form $\gamma + 3c + (2 + i \sqrt{5})b$.
    When quotienting with $(3, 2 + i \sqrt{5})$, everything except $\gamma$ is absorbed away.
    Therefore, $\mathbb{Z}[i \sqrt{5}] \big/ (3, 2 + i \sqrt{5}) \cong \mathbb{Z}_{3}$.
  \item \emph{Example 2:} Consider the ring $\mathbb{Z}$ and ideal $(m)$.
    Then $\mathbb{Z} \big/ (m) \cong \mathbb{Z}_{m}$.
    If we consider $(p)$ such that $p$ is prime, then $(p)$ is maximal, the proof of which follows from the fact that the $gcd$ of any two numbers is their integer linear combination.
    This gives us the fact that $\mathbb{Z} \big/ (p)$ is a field, and is denoted by $\mathbb{F}_{p}$.
  \item \emph{Example 3:} $F[x]$ where $F$ is a ring.
    All ideals can be shown to be principal.
    Consider a polynomial $q(x)$.
    $F[x] \big/ (q(x))$ is the ring of all polynomials of degree less than the degree of $q(x)$, with operations done modulo $q(x)$.
    If $q(x)$ is irreducible, then $(q(x))$ is maximal and the quotient ring is a field.
    \par
    In particular, if $F[x]$ is $\mathbb{R}[x]$ and $q(x) = x^{2} + 1$, then the quotient field is isomorphic to $\mathbb{C}$.
  \item \emph{Example 4:} Take $F[x, y] \big/ (x^{2} + y^{2} + 1)$.
    $(x^{2} + y^{2} + 1)$ is a principal ideal.
    Consider two elements that belong to the same class, $q_{1}, q_{2}$.
    $q_{1} - q_{2} \in (x^{2} + y^{2} + 1) \Rightarrow q_{1} - q_{2} = (x^{2} + y^{2} + 1) * p(x, y)$ for some polynomial $p$.
    This implies that $q_{1}, q_{2}$ take on the same values on the circle $x^{2} + y^{2} + 1$.
    Put another way, the quotient group captures polynomials that are the same over the given curve, here the circle.
\end{itemize}

\subsection{Prime and Irreducible Ideals}
\begin{definition} \label{def:idealprime}
  Ideal $I$ of ring $R$ is \textbf{prime} if for every $a, b \in R$, if $a * b \in I$, then $a \in I$ or $b \in I$.
\end{definition}
\begin{definition} \label{def:idealirred}
  Ideal $I$ of ring $R$ is \textbf{irreducible} if for every pair of ideals $I_{1}, I_{2}$ of $R$, if $I = I_{1} * I_{2}$ then either $I_{1} = (1)$ or $I_{2} = (1)$.
\end{definition}
\begin{lemma} \label{lem:primeintegral}
  Let $R$ be a ring, and $I$ a prime ideal.
  Then $R \big/ I$ is an integral domain.
\end{lemma}
\begin{proof} \label{proof:primeintegral}
  Let $a + I, b + I \in R \big/ I$ such that $(a + I) * (b + I) = I \Rightarrow ab + I = I \Rightarrow ab \in I$.
  This implies that either $a$ or $b$ is in $I$, which means one of $(a + I), (b + I)$ is zero.
\end{proof}
